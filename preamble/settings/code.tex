%%%%%%%%%%%%%%%%%%%%%%%%%%%%%%%%
%            Code              %
%%%%%%%%%%%%%%%%%%%%%%%%%%%%%%%%
\newcommand{\code}[1]{{\sf #1}}                    % \code{X} writes X in a code-appropriate font




%%%%%%%%%%%%%%%%%%%%%%%%%%%%%%%%
%         lstlisting           %
%%%%%%%%%%%%%%%%%%%%%%%%%%%%%%%%

% Import lstlistings - beautiful sourcecode!
\usepackage{listings}


% Custom language definitions
% Definition of Pseudocode
\lstdefinelanguage{pseudocode}{
  keywords=[1]{
           break, break, by, do, downto, else, elif, error, for, if, let, repeat, return, then, to, until, while, while
      },                                           % list of keywords, first and last are not used for some stupid reason
  keywords=[2]{
        and, and, or, NIL, NIL
  }
  sensitive=false,                                 % keywords are not case-sensitive
  morecomment=[l]{//},                             % l is for line comment
  morecomment=[s]{/*}{*/},                         % s is for start and end delimiter
  morestring=[b]"                                  % strings are enclosed in double quotes
}


% lstlisting - General settings
\lstset{
  language=pseudocode,                             % choose language
  escapeinside={*@}{@*},                           % if you want to add LaTeX within your code
  literate={æ}{{\ae}}1{ø}{{\oe}}1{å}{{\aa}}1       % allow æ, ø and å in code
           {Æ}{{\AE}}1{Ø}{{\O}}1{Å}{{\AA}}1,       %     (this change was taken from the preamble of the MatFysTutor LaTeX Guide)
}

% lstlisting - Whitespace
\lstset{
  showspaces=false,                                % show spaces everywhere - adding particular underscores
  showstringspaces=false,                          % underline spaces within strings only.
  showtabs=false,                                  % show tabs within strings adding particular underscores.
  % breaklines=true,                                 % automatically break lines
  % breakatwhitespace=true,                          % automatically break should there only be white space.
  tabsize=4                                        % tab width
}

% lstlisting - Colors, font and styling
\lstset{
  stepnumber=1,                                       % step between to line-numbers. 1 = each line is numbered
  numbers=left,                                       % numbering: none, left, right
  numbersep=5pt,                                      % distance between linenumbers and code
  numberstyle=\color{lstComment},                     % change style of numbering - currently grey.
  columns=[c]fixed,                                   % makes it monospaced
  basicstyle=\ttfamily \color{lstBase},  % set basic color
  commentstyle=\color{lstComment},                    % set color of comments
  keywordstyle=[1]\color{lstKey},                     % set color of primary keywords
  keywordstyle=[2]\color{lstKey2},                    % set color of secondary keywords
  stringstyle=\color{lstString},                      % set color of strings
}

% lstlisting - Put it beautifully in the middle
% Currently off, since captions are misbehaving
% \lstset{
%   xleftmargin= .1\textwidth,                       % leftmargin being 10% of the current width
%   xrightmargin= .1\textwidth,                      % right margin also 10%
% }

% lstlisting - Minimalistic borders and captions
\lstset{
  frame=top,                                       % bar on top
  frame=bottom,                                    % bar on bottom
  captionpos=b                                     % caption at the bottom
}
\DeclareCaptionFormat{listing}{\makebox[0.5in][l]{#1#2}\parbox[t]{\dimexpr \captionwidth-0.5in}{#3}\hline}
\captionsetup[lstlisting]{format=listing, singlelinecheck=off, labelsep=colon}
